%\documentclass[12pt]{article}
\documentclass[12pt]{scrartcl}
\title{Note01}
\nonstopmode
%\usepackage[utf-8]{inputenc}
\usepackage{graphicx} % Required for including pictures
\usepackage[figurename=Figure]{caption}
\usepackage{float}    % For tables and other floats
\usepackage{verbatim} % For comments and other
\usepackage{amsmath}  % For math
\usepackage{amssymb}  % For more math
\usepackage{fullpage} % Set margins and place page numbers at bottom center
\usepackage{paralist} % paragraph spacing
\usepackage{listings} % For source code
\usepackage{subfig}   % For subfigures
%\usepackage{physics}  % for simplified dv, and 
\usepackage{enumitem} % useful for itemization
\usepackage{siunitx}  % standardization of si units

\usepackage{tikz,bm} % Useful for drawing plots
%\usepackage{tikz-3dplot}
\usepackage{circuitikz}
\usepackage[UTF8]{ctex}


\begin{document}

\begin{center}
	\hrule
	\vspace{.4cm}
	{\textbf { \large 云峰朋辈辅学微甲提升2组 --- 第8讲}}
\end{center}
{\textbf{内容提要:}\ 曲线积分 \hspace{\fill} \textbf{Date:} May 29 2022    \\
{ \textbf{主讲人:}} \ Famiglisti @CC98  \hspace{\fill} \textbf{Place:} 碧2党员之家 \\
	\hrule
~\\


\paragraph*{\large 1 知识概要}\leavevmode \newline

\begin{enumerate}
    \item \textbf{第一类曲线积分————对弧长的曲线积分}\\
    $\int_L f(x,y)ds=\lim_{\lambda  \to 0}\sum_{i = 1}^{n}f(\xi _i,\eta _i)\Delta s_i  $\\
    $\int_L f(x,y,z)ds=\lim_{\lambda  \to 0}\sum_{i = 1}^{n}f(\xi _i,\eta _i,\varsigma _i)\Delta s_i  $\\
    积分可视化:
    \begin{figure}[H]
        \centering
        \includegraphics[width=10cm]{1.png}
        \caption{积分值为红色区域面积}
        \end{figure}
  $\int_L ds=l  $,l为曲线长度
    \item \textbf{第二类曲线积分————对坐标的曲线积分}
    \begin{enumerate}
        \item 二维空间:$\int _LP(x,y)dx+Q(x,y)dy$
        \item 三维空间:$\int _LP(x,y,z)dx+Q(x,y,z)dy+R(x,y,z)dz$
    \end{enumerate}
    $\int _LP(x,y)dx+Q(x,y)dy=\int _L(P(x,y)\cos \alpha+Q(x,y)\cos \beta) ds$\\
    $\int _LP(x,y,z)dx+Q(x,y,z)dy+R(x,y,z)dz=\int _L(P(x,y,z)\cos \alpha+Q(x,y,z)\cos \beta+R(x,y,z)\cos \gamma ) ds$
    \item \textbf{Green公式}\\
    $\iint_D(-\frac{\partial P}{\partial y}+\frac{\partial Q}{\partial x})dxdy=\int _{\partial D}Pdx+Qdy$\\
    使用条件:
    \begin{itemize}
        \item D为$\mathbb{R} ^2$内的一个有界闭区域
        \item $\partial D$由光滑曲线或逐段光滑曲线组成
        \item 函数P(x,y),Q(x,y)在D内有关于自变量x,y的连续偏导数
        \item $\partial D$的方向关于D是正向的
    \end{itemize}
    推论:由逐段光滑的简单曲线C所界的面积S可用曲线积分表示为$S=\oint _Cxdy=-\oint _Cydx=\frac{1}{2}\oint _Cxdy-ydx$
\end{enumerate}



\paragraph*{\large 2 真题}\leavevmode \newline
【18-19final】设曲线C为一元函数$y=\int _1^x\sqrt{sin(t^2)}dt,x\in[1,\sqrt{\frac{\pi}{2}}]$
的图像,试计算第一类曲线积分$\int _Cxds$
\\
\\
\\
\\
\\
\\
\\
\\
\\
【18-19final】设$\gamma $为圆柱螺线的一段$x=\cos t,y=\sin t,z=2t,0\leq t\leq \pi$
其中$\gamma $的正向为参数t增加的方向,并设$\gamma_1 $为$\gamma $的前半段有向弧($t \in[0,\frac{\pi}{2}]$),
其正向与$\gamma $一致,又设L为圆柱螺线$\gamma $上点$(0,1,\pi)$处的切线,并设$L_1$
为切线L上一段以点$(0,1,\pi)$为起点,正向与$\gamma $正向一致,长度为$\sqrt{5}$的有向直线段,试计算第二类曲线积分
$\int_{\gamma _1\cup L_1}yzdx+xzdy+xydz $
\\
\\
\\
\\
\\
\\
\\
\\
\\
【18-19final】设$D=\left\{(x,y)\in\mathbb{R} ^2|0\leq x \leq 1,y_1(x)\leq y \leq y_2(x) \right\} $,
其中$y=y_1(x),y=y_2(x)$是[0,1]上的连续函数,u(x,y)在包含D的一个开集上
有连续的一阶偏导函数,设D的边界$\partial D$的正向为逆时针方向\\
证明:$\oint _{\partial D}udx=-\iint \frac{\partial u}{\partial y}dxdy$
\\
\\
\\
\\
\\
\\
\\
\\
\\
【19-20final】设二元函数P(x,y),Q(x,y)在平面但联通区域D上有所有连续的一阶偏导函数,且在D内的任意一条
光滑的简单封闭曲线C上,都有$\oint _{C^+}P(x,y)dy-Q(x,y)dx=0$成立。试证明在D内有$\frac{\partial P}{\partial x}+\frac{\partial Q}{\partial y}=0$恒成立 
\\
\\
\\
\\
\\
\\
\\
\\
\\
【20-21final】 设空间曲线C为$\left\{ (x,y,z)\in \mathbb{R} ^3|x=t,y=\frac{t^2}{2},z=\frac{t^3}{3},t \in [0,e] \right\} $,
计算第一类曲线积分
\begin{align}
    I_1=\int _C \frac{xz}{\sqrt{1+2y+4y^2}}ds \notag
\end{align}
\\
\\
\\
\\
\\


\paragraph*{\large 3 例题}\leavevmode \newline
【example 1】计算积分
\begin{align}
    I=\oint _C\frac{\cos(\vec{r},\vec{n})}{r}ds \notag
\end{align}
其中C为逐段光滑的简单闭曲线,(0,0)在C外,$\vec{r}=(x,y)$,$r=|\vec{r}|=\sqrt{x^2+y^2}$,$\vec{n}$为
C上的单位外法向量。\\
思考:(0,0)在C内,(0,0)在C上的情形
\\
\\
\\
\\
\\
\\
\\
\\
\\
\\
\\
\\
\\
【example 2】计算
\begin{align}
    I=\oint _C\frac{e^y}{x^2+y^2}[(x\sin x+y\cos x)dx+(y\sin x-x\cos x)dy]\notag
\end{align}
其中$C:x^2+y^2=1$,取逆时针方向
\\
\\
\\
\\
\\
\\
\\
\\
\\
\\
\\
\\
\\
【example 3】设a,b,c为常数,满足$ac-b^2>0$,
\begin{align}
    w=\frac{xdy-ydx}{ax^2+2bxy+cy^2} \notag
\end{align}
求w关于原点(0,0)的循环常数$\oint _C w$,其中C可取围绕(0,0)的任一简单封闭曲线,并取逆时针方向为正向
\\
\\
\\
\\
\\
\\
\\
\\
\\
\\
\\
\\
\\
\\
\\
【example 4】计算  $x^2+y^2+z^2=a^2$在第一卦限部分的边界的质心坐标
\\
\\
\\
\\
\\
\\
\\
\\
\newpage
\paragraph*{\large 4 参考书目}\leavevmode \newline
\begin{enumerate}
    \item 陈纪修,於崇华,金路. 数学分析.下册[M]. 北京:高等教育出版社,2019.5
    \item 汤家凤.考研数学复习大全[M]. 北京:中国原子能出版社,2019.2
    \item 谢惠民.数学分析习题课讲义.下册[M]. 北京:高等教育出版社,2004.1
    \item 苏德矿,吴明华.微积分.下[M].北京:高等教育出版社,2007.7
\end{enumerate}
\end{document}