
%\documentclass[12pt]{article}
\documentclass[12pt]{scrartcl}
\title{Note01}
\nonstopmode
%\usepackage[utf-8]{inputenc}
\usepackage{graphicx} % Required for including pictures
\usepackage[figurename=Figure]{caption}
\usepackage{float}    % For tables and other floats
\usepackage{verbatim} % For comments and other
\usepackage{amsmath}  % For math
\usepackage{amssymb}  % For more math
\usepackage{fullpage} % Set margins and place page numbers at bottom center
\usepackage{paralist} % paragraph spacing
\usepackage{listings} % For source code
\usepackage{subfig}   % For subfigures
%\usepackage{physics}  % for simplified dv, and 
\usepackage{enumitem} % useful for itemization
\usepackage{siunitx}  % standardization of si units

\usepackage{tikz,bm} % Useful for drawing plots
%\usepackage{tikz-3dplot}
\usepackage{circuitikz}
\usepackage[UTF8]{ctex}

%%% Colours used in field vectors and propagation direction
\definecolor{mycolor}{rgb}{1,0.2,0.3}
\definecolor{brightgreen}{rgb}{0.4, 1.0, 0.0}
\definecolor{britishracinggreen}{rgb}{0.0, 0.26, 0.15}
\definecolor{cadmiumgreen}{rgb}{0.0, 0.42, 0.24}
\definecolor{ceruleanblue}{rgb}{0.16, 0.32, 0.75}
\definecolor{darkelectricblue}{rgb}{0.33, 0.41, 0.47}
\definecolor{darkpowderblue}{rgb}{0.0, 0.2, 0.6}
\definecolor{darktangerine}{rgb}{1.0, 0.66, 0.07}
\definecolor{emerald}{rgb}{0.31, 0.78, 0.47}
\definecolor{palatinatepurple}{rgb}{0.41, 0.16, 0.38}
\definecolor{pastelviolet}{rgb}{0.8, 0.6, 0.79}
\begin{document}

\begin{center}
	\hrule
	\vspace{.4cm}
	{\textbf { \large 云峰朋辈辅学微甲提升2组 --- 第2讲}}
\end{center}
{\textbf{内容提要:}\ 幂级数 \hspace{\fill} \textbf{Date:} March 27 2022    \\
{ \textbf{主讲人:}} \ Famiglisti @CC98  \hspace{\fill} \textbf{Place:} 碧2党员之家 \\
	\hrule

    
\paragraph*{\large 1 知识概要}
\begin{enumerate}
    \item  函数项级数的定义
    \item  幂级数基本定理——Abel定理
    \item  求收敛半径与收敛域的两种基本方法(根值、比值)
    \item  幂级数的分析性质(连续性、逐项可导性、逐项可积性、收敛半径不变性)
    \item  绝对收敛与条件收敛
\end{enumerate}

\paragraph*{\large 2 基本题型}

\paragraph*{题型一:求幂级数的收敛半径和收敛域}\leavevmode \newline
根值法、比值法、换元法\newline
【example 1】 求 $\sum_{n = 1}^{\infty}\frac{1}{2n+1} (\frac{1-x}{1+x})^n$的收敛域\\
\\
\\
\\
\\
\paragraph*{题型二:幂级数求和函数} \leavevmode \newline
别忘了求收敛半径和收敛域!\leavevmode \newline
\textbf{情形一:}求$\sum_{n = 0}^{\infty} P(n)x^n $的和函数,
其中P(n)为n的多项式,求和函数常用工具
\begin{enumerate}[label={\alph*}] 
    \item 级数的逐项可积性
    \item $\sum_{n = 0}^{\infty}x^n=\frac{1}{1-x}(-1<x<1)  $
    \item $\sum_{n = 0}^{\infty}(-1)^nx^n=\frac{1}{1+x}(-1<x<1)  $
\end{enumerate}
【example1】$\sum_{n = 1}^{\infty}\frac{n}{2^{n+1}} (x-1)^{2n+1}$\newline
\\
\\
\\
\\
\\
\\
【example2】 $\sum_{n = 0}^{\infty}n^2x^n$\leavevmode \newline
\\
\\
\\
\\
\\
\\

\noindent\textbf{情形二:}求$\sum_{n = 0}^{\infty} \frac{x^n}{ P(n)} $的和函数,
其中P(n)为n的多项式,求和函数常用工具
\begin{enumerate}[label={\alph*}] 
    \item 级数的逐项可导性
    \item $\sum_{n = 1}^{\infty}\frac{(-1)^{n-1}x^n}{n} =\ln (1+x) (-1<x\leqslant 1)  $
    \item $\sum_{n = 1}^{\infty}\frac{x^n}{n} =-\ln (1-x) (-1\leqslant x<1)  $
\end{enumerate}
【example1】$\sum_{n = 1}^{\infty}\frac{x^{n+1}}{n(n+1)2^n} $\leavevmode \newline
\\
\\
\\
\\
\\
\\

\noindent【example2】$\sum_{n = 1}^{\infty}\frac{(-1)^nx^{2n-1}}{n(2n-1)} $\leavevmode \newline
\\
\\
\\
\\
\\
\\

\noindent\textbf{情形三:}幂级数系数的分母中含n!,求和函数常用工具
\begin{enumerate}[label={\alph*}] 
    \item $\sum_{n = 0}^{\infty}\frac{x^n}{n!} =e^x(-\infty <x<+\infty )  $
    \item $\sum_{n = 0}^{\infty}\frac{(-1)^nx^{2n}}{2n!} =\cos x(-\infty <x<+\infty )  $
    \item $\sum_{n = 0}^{\infty}\frac{(-1)^nx^{2n+1}}{(2n+1)!} =\sin  x(-\infty <x<+\infty )  $
    \item 求和函数满足的微分方程
\end{enumerate}
【example1】$\sum_{n = 0}^{\infty}\frac{n^2+1}{2^nn!} x^n$
\\
\\
\\
\\
\\
【example2】$\sum_{n = 0}^{\infty}\frac{(2n)!!}{(2n+1)!!} x^{2n+1}$ (-1<x<1)
\\
\\
\\
\\
\\
\textbf{情形四:}求$\sum_{n = 0}^{\infty} P(n)x^n $的和函数,其中P(n)为复杂分式,常通过换元等方法将分式
化为标准多项式或标准分式\leavevmode \newline
【example1】$\sum_{n = 0}^{\infty}\frac{(n-1)^2}{n+1} x^n$
\\
\\
\\
\\
\\
\paragraph*{题型三:函数展成幂级数} \leavevmode \newline
积分法、微分法、裂项法、熟记常用级数展开公式\leavevmode \newline
【example1】将$f(x)=\arctan \frac{1+x}{1-x} $展成x的幂级数
\\
\\
\\
\\
\\
【example2】求$\ln \frac{\sin x}{x}$在x=0的幂级数展开(到$x^4$)\leavevmode \newline 
\\
\\
\\
\\
\\
\paragraph*{题型四:特殊的常数项级数的求和}\leavevmode \newline
将数项级数中的$a^n$代换为$x^n$,并求和函数\leavevmode \newline
【example1】$\sum_{n = 2}^{\infty}\frac{1}{2^n(n^2-1)}$
\\
\\
\\
\\
\\
\\
\paragraph*{\large 3 真题解析}\leavevmode \newline
【18-19mid】判断级数$\sum_{n = 1}^{\infty} \frac{1}{1+\sqrt{2}+\cdots +\sqrt{n} }$ 的敛散性 \leavevmode \newline
\\
\\
\\
\\
\\
\\
\\
【18-19mid】将$f(x)=\arcsin x$展开成x的幂级数,并求$f^{(99)}(0)$\leavevmode \newline
\\
\\
\\
\\
\\
\\
\\
【18-19mid】设$0<a_0<1,a_{n+1}=a_{n}(1-a_{n})(n=0,1,2,\cdots )$,证明
\begin{enumerate}
    \item $\sum_{n = 0}^{\infty} (-1)^na_n$收敛 
    \item $\sum_{n = 0}^{\infty} a_n$发散
\end{enumerate}\leavevmode \newline
\\
\\
\\
\\
\\
\\
【18-19final】求幂级数$\sum_{n = 2}^{\infty}(-1)^n\frac{x^n}{n+1}$的收敛半径与和函数\leavevmode \newline
\\
\\
\\
\\
\\
\\
【19-20final】求级数$\sum_{n = 1}^{\infty} (-1)^n\frac{\alpha (\alpha -1)\cdots (\alpha -n+1)}{(2^n+3^n)n!}$ 的收敛半径\leavevmode \newline
\\
\\
\\
\\
\\
\\
\end{document}