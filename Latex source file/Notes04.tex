
%\documentclass[12pt]{article}
\documentclass[12pt]{scrartcl}
\title{Note01}
\nonstopmode
%\usepackage[utf-8]{inputenc}
\usepackage{graphicx} % Required for including pictures
\usepackage[figurename=Figure]{caption}
\usepackage{float}    % For tables and other floats
\usepackage{verbatim} % For comments and other
\usepackage{amsmath}  % For math
\usepackage{amssymb}  % For more math
\usepackage{fullpage} % Set margins and place page numbers at bottom center
\usepackage{paralist} % paragraph spacing
\usepackage{listings} % For source code
\usepackage{subfig}   % For subfigures
%\usepackage{physics}  % for simplified dv, and 
\usepackage{enumitem} % useful for itemization
\usepackage{siunitx}  % standardization of si units

\usepackage{tikz,bm} % Useful for drawing plots
%\usepackage{tikz-3dplot}
\usepackage{circuitikz}
\usepackage[UTF8]{ctex}

%%% Colours used in field vectors and propagation direction
\definecolor{mycolor}{rgb}{1,0.2,0.3}
\definecolor{brightgreen}{rgb}{0.4, 1.0, 0.0}
\definecolor{britishracinggreen}{rgb}{0.0, 0.26, 0.15}
\definecolor{cadmiumgreen}{rgb}{0.0, 0.42, 0.24}
\definecolor{ceruleanblue}{rgb}{0.16, 0.32, 0.75}
\definecolor{darkelectricblue}{rgb}{0.33, 0.41, 0.47}
\definecolor{darkpowderblue}{rgb}{0.0, 0.2, 0.6}
\definecolor{darktangerine}{rgb}{1.0, 0.66, 0.07}
\definecolor{emerald}{rgb}{0.31, 0.78, 0.47}
\definecolor{palatinatepurple}{rgb}{0.41, 0.16, 0.38}
\definecolor{pastelviolet}{rgb}{0.8, 0.6, 0.79}
\begin{document}

\begin{center}
	\hrule
	\vspace{.4cm}
	{\textbf { \large 云峰朋辈辅学微甲提升2组 --- 第4讲}}
\end{center}
{\textbf{内容提要:}\ 空间解析几何、多元函数微分学 \hspace{\fill} \textbf{Date:} April 10 2022    \\
{ \textbf{主讲人:}} \ Famiglisti @CC98  \hspace{\fill} \textbf{Place:} 碧2党员之家 \\
	\hrule
~\\
\begin{center}
\textbf{\large{Part A :空间解析几何} }
\end{center}  
\paragraph*{\large 1 知识概要}
\begin{enumerate}
    \item  \textbf{混合积运算性质}
    \begin{enumerate}
        \item \textbf{a,b,c}共面$\Leftrightarrow $(\textbf{a,b,c})=0
        \item \textbf{a,b,c}=\textbf{b,c,a}=\textbf{c,a,b}
        \item $\vec{a}\times (\vec{b} \times \vec{c})= (\vec{a}\cdot \vec{c})b
        -(\vec{a}\cdot \vec{b})\vec{c}$
        \item Lagrouge恒等式$(\vec{a}\times \vec{b})\cdot (\vec{c}\times \vec{d})= 
        (\vec{a}\cdot \vec{c})(\vec{b}\cdot \vec{d})-
        (\vec{a}\cdot \vec{d})(\vec{b}\cdot \vec{c})$
        \item $(\vec{a}\times \vec{b})\times (\vec{c}\times \vec{d})=
         (\vec{a},\vec{b},\vec{d})\vec{c}- 
         (\vec{a},\vec{b},\vec{c})\vec{d}=
         (\vec{a},\vec{c},\vec{d})\vec{b}-
         (\vec{b},\vec{c},\vec{d})\vec{a}
        $
        \item $
        (\vec{a}\times \vec{b},\vec{c}\times \vec{d},\vec{e}\times \vec{f})
        = (\vec{a},\vec{b},\vec{d}) (\vec{c},\vec{e},\vec{f})-
        (\vec{a},\vec{b},\vec{c}) (\vec{d},\vec{e},\vec{f})
        $
    \end{enumerate}\leavevmode \newline
    \item  \textbf{几个距离公式}
    \begin{enumerate}
        \item 点到平面:$d=\frac{\left\lvert Ax_0+By_0+Cz_0+D \right\rvert }
        {\sqrt{A^2+B^2+C^2}} $
        \begin{figure}[htbp]
            \centering
            \includegraphics[width=4cm]{t3.png}
            \end{figure}
        \item 点到直线:$d=\frac{\left\lvert  \vec{M_0M} \times\vec{s} \right\rvert}
        {\left\lvert \vec{s}\right\rvert } $
        \begin{figure}[htbp]
            \centering
            \includegraphics[width=4cm]{t4.png}
            \end{figure}
        \item 两平行平面:$d=\frac{\left\lvert D_2-D_1 \right\rvert }
        {\sqrt{A^2+B^2+C^2}} $
        \item 两异面直线:$d=\frac{\left\lvert \vec{M_1M_2}\cdot(\vec{v_1}\times\vec{v_2})
         \right\rvert }{\left\lvert \vec{v_1}\times\vec{v_2} \right\rvert } $
    \end{enumerate}\leavevmode \newline
    \item  \textbf{平面束方程及应用}\\
    \hspace*{0.7cm}设L:$\left\{
        \begin{array}{lr}
        A_1x+B_1y+C_1z+D_1=0 &  \\
        A_2x+B_2y+C_2z+D_2=0 &  
        \end{array}
        \right.$为一条直线,则过L的所有平面成为过直线L的平面束,平面束方程为
        \begin{gather}
            \pi':A_1x+B_1y+C_1z+D_1+\lambda ( A_2x+B_2y+C_2z+D_2)=0\notag
        \end{gather}
        \hspace*{0.7cm}对于空间直线和平面的问题,解题思路比较灵活,方法往往不唯一,通过对比发现,利用
        平面束方程求解,思路更加清晰,过程更简洁。(见例题1、真题1)
    \item  \textbf{曲面}
    \begin{enumerate}
        \item 柱面
        \item 旋转曲面
        \begin{enumerate}
            \item 二维空间曲线的旋转曲面
            \item 三维空间直线的旋转曲面(见真题2)
        \end{enumerate}
    \end{enumerate}
\end{enumerate}\leavevmode \newline

\paragraph*{\large 2 例题}\leavevmode \newline
【example 1】求直线L
$\left\{
\begin{array}{lr}
2x-y+z=1 &  \\
x+y-z=-1 &  
\end{array}
\right.$
在平面$\pi:x+2y-z=0$上的投影直线的方程    \leavevmode \newline

\newpage
\paragraph*{\large 3 真题解析}\leavevmode \newline
【18-19mid】求过点A(-1,0,4)且平行于平面3x-4y+z=0,又与直线$\frac{x+1}{1}=\frac{y-1}{3}=\frac{z}{2}   $
相交的直线方程 \leavevmode \newline
\\
\\
\\
\\
\\
\\
\\
【18-19mid】求直线L
            $\left\{
            \begin{array}{lr}
            x+y+z=0 &  \\
            y-z-1=0 &  
            \end{array}
            \right.$
绕Oz轴旋转所成的旋转曲面方程    \leavevmode \newline
\\
\\
\\
\\
\\
\\
\\
【18-19final】设有二次曲面$S:x^2+xy+y^2-z^2=1$,试求曲面S上点(1,-1,0)
处的切平面$\pi$的平面方程。\leavevmode \newline
\\
\\
\\
\\
\\
\\
\\
\\
\\
【20-21final】求曲面$S:z=x^2y^3-e^z+e$上点(1,1,1)处的切平面
方程及法线方程
\leavevmode \newline  
\\
\\
\newpage
\begin{center}
    \textbf{\large{Part B :多元函数微分学} }
    \end{center} 
\paragraph*{\large 1 知识概要}
\begin{enumerate}
    \item  \textbf{点集拓扑学的一些术语}
    \hspace*{0.7cm} 内点、外点、边界、开集、闭集、连通
    \item  \textbf{连续、可偏导、可微}(以二元函数为例)
    \begin{enumerate}
        \item 连续:$\lim_{x \to x_0,y\to y_0}f(x,y)=f(x_0,y_0)$
        ,称f(x,y)在$(x_0,y_0)$处连续
        \item 可偏导:$\lim_{\Delta x \to 0}
        \frac{f(x_0+\Delta x,y_0)-f(x_0,y_0)}{\Delta x}$存在,
        称f(x,y)在$(x_0,y_0)$处对x可偏导
        \item 可微:若z=f(x,y)在(x,y)处全增量
        $\Delta z=f(x+\Delta x,y+\Delta y)-f(x,y)$可表示为
        \begin{gather}
            \Delta z=A\Delta x+B\Delta y+o(\rho ) \quad
            (\rho=\sqrt{\Delta x^2+\Delta y^2}\rightarrow 0 )\notag\\
            A=f_x'(x,y),\quad B=f_y'(x,y)\notag
        \end{gather}
        则称函数f(x,y)在点(x,y)处可微,其中全微分$dz=A\Delta x+B\Delta y$
        \item 关系图解
        \begin{figure}[htbp]
            \centering
            \includegraphics[width=10cm]{t1.jpg}
            \end{figure}
    \end{enumerate}
    \item  \textbf{偏导数计算法则}
    \begin{enumerate}
        \item 复合函数:链导法则
        \begin{figure}[htbp]
            \centering
            \includegraphics[width=12cm]{t2.png}
            \end{figure}
        \item 隐函数
        \begin{enumerate}
            \item 由一个方程确定的隐函数
            \item 由多个方程确定的隐函数——Jacobi行列式法
        \end{enumerate}
    \end{enumerate}

    
\end{enumerate}

\paragraph*{\large 2 例题}\leavevmode \newline
【example 1】设$f(x,y)=\left\{
    \begin{array}{lr}
    (x^2+y^2)\sin {\frac{1}{x^2+y^2}},(x,y)\neq (0,0) &  \\
    0,(x,y)= (0,0) &  
    \end{array}
    \right.$
    研究函数f(x,y)在(0,0)处的连续性、可偏导性、可微性及一阶偏导的连续性。
\\
\\
\\
\\
\\
\\
\\
\paragraph*{\large 3 真题解析}\leavevmode \newline
【18-19mid】已知$z=(e^x+y^2)^y$,求$\frac{\partial z}{\partial x},
\frac{\partial z }{\partial y}$ \leavevmode \newline
\\
\\
\\
\\
\\
\\
\\
【18-19mid】设$u=u(x,y)$对新变量$\xi,\eta $具有二阶偏导数
,求a,使得在变换$\xi =x+ay,\eta =x-y$下,将方程
$\frac{\partial ^2 u}{\partial x^2}+
4\frac{\partial ^2 u}{\partial x\partial y}
+3\frac{\partial ^2 u}{\partial y^2}=0  $简化为
$\frac{\partial ^2 u}{\partial \xi \partial \eta }=0 $     \leavevmode \newline
\\
\\
\\
\\
\\
\\
\\
【18-19final】设f(x,y)在$\mathbb{R} ^2$上有连续的一阶偏导数,
且$\forall t>0,\forall(x,y)\in\mathbb{R} ^2,f(tx,ty)=t^3f(x,y) $
证明:
$\forall(x,y)\in\mathbb{R} ^2,x\frac{\partial f}{\partial x}(x,y)+
y\frac{\partial f}{\partial y}(x,y)=3f(x,y) $\leavevmode \newline
\\
\\
\\
\\
\\
\\
\\
\\
\\
\\
【18-19final】设$f(x,y)=(xy)^{\frac{2}{3}}$,
\begin{enumerate}
    \item 求$\frac{\partial f}{\partial x}(0,0),
    \frac{\partial f}{\partial y}(0,0)$
    \item 证明:f在点(0,0)处可微
\end{enumerate}\leavevmode \newline
\\
\\
\\
\\
\\
\\
\\
\\
\\
【19-20final】设
$f(x,y)=\left\{
            \begin{array}{lr}
            \frac{xy^3}{x^2+y^2},(x,y)\neq (0,0) &  \\
            0,(x,y)= (0,0) &  
            \end{array}
            \right.$
试求$\frac{\partial f}{\partial x}(x,y),
\frac{\partial^2 f}{\partial x \partial y}(0,0)$\leavevmode \newline
\\
\\
\\
\\
\\
\\
\\
\\
\\
【19-20final】证明:$\forall x\in(0,1),y\in(0,+\infty) $
,不等式$y(1-x)x^{y+1}<e^{-1}$成立.\leavevmode \newline
\\
\\
\\
\\
\\
\\
\\
\\
\\
\\
\\
【20-21final】设z=f(u,v)在平面上可微,且满足
\begin{gather}
    \forall x \in \mathbb{R} ,f(x^2+1,e^x)=e^{(x+1)^2},
    f(x^2,x)=x^2e^{x^2} \notag
\end{gather}
求f在点(1,1)处的全微分$df| _{(1,1)}$\leavevmode \newline
\\
\\
\\
\\
\\
\\
\\
\\
【20-21final】设D是平面上的一个有界闭区域,z=z(x,y)在D上连续,
在$D^o$上有所有的连续二阶偏导函数,且满足
$\forall(x,y)\in D^o,
\frac{\partial^2 z}{\partial x^2}(x,y)+
\frac{\partial^2 z}{\partial y^2}(x,y)=0,
\frac{\partial^2 z}{\partial x \partial y}(x,y)\neq 0$\\
证明:z(x,y)在D上的最值只能在D的边界上取到。
\leavevmode \newline
\\
\\
\\
\\
\\
\\
\\
\paragraph*{\large 4 参考文献}\leavevmode \newline
\begin{enumerate}
    \item 陈纪修,於崇华,金路. 数学分析.下册[M]. 北京:高等教育出版社,2019.5
    \item 汤家凤.考研数学复习大全[M]. 北京:中国原子能出版社,2019.2
    \item 谢惠民.数学分析习题课讲义.下册[M]. 北京:高等教育出版社,2004.1
    \item 张艳敏,申子慧.平面束方程的详细证明及应用举例[J].湖南理工学院学报(自然科学版),2016,29(04):20-23.DOI:10.16740/j.cnki.cn43-1421/n.2016.04.004.
    \item 苏德矿,吴明华.微积分.下[M].北京:高等教育出版社,2007.7
    \item 丘维声.解析几何.3版[M].北京:北京大学出版社,2015.7
\end{enumerate}
 
\end{document}