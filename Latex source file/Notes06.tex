\documentclass[12pt]{ctexbook}
\usepackage{tikz}
\usepackage{graphicx}
\usepackage{subfig}
\usepackage{float}
\usepackage{amsmath}
\usepackage{amssymb}
\usepackage{booktabs}
\usepackage{xcolor}
\usepackage[hidelinks,linkcolor=black]{hyperref}
\usepackage{titletoc}%目录
\usepackage[hmargin=21mm,vmargin=23mm]{geometry}
\usepackage{fancyhdr}%页眉页脚
\usepackage{caption}
\usepackage{pifont}%提供带圈数字
\usepackage{enumitem}
\usepackage{lscape}
\usepackage[amsmath]{ntheorem}

\definecolor{colorAll}{RGB}{204, 255, 0}
\definecolor{colorLine}{RGB}{255, 153, 0}
\definecolor{colorWord}{RGB}{204, 51, 51}
\definecolor{colorBall}{RGB}{51, 153, 204}
\definecolor{colorCircle}{RGB}{0, 255, 0}
\definecolor{colorBg}{RGB}{255, 255, 255}

%页面背景颜色
\pagecolor{colorBg}

% 设置页眉线
\renewcommand{\headrulewidth}{0.6pt}
\ctexset{chapter={pagestyle=fancy}}%使每一章首页都有页眉

%设置目录格式
\setcounter{tocdepth}{2}
\titlecontents{chapter}%章
              [4.5em]%
              {\addvspace{2.3mm}\bf}%
              {\contentslabel{4.5em}}%
              {}%
              {\titlerule*[5pt]{$\cdot$}\contentspage}%
\titlecontents{section}%节
              [4.5em]%
              {}%
              {\contentslabel{3.3em}}%
              {}%
              {\titlerule*[5pt]{$\cdot$}\contentspage}%

% 设置章节标题格式
\ctexset {
  chapter = {%章标题格式
    beforeskip = 20pt,
    fixskip = true,
    format += \zihao{-2},
    number = \arabic{chapter},
    afterskip = 20pt,
  },
  section = {%节标题格式
    name = {\S~}, 
    format += \zihao{-3}\raggedright,
  },
  contentsname = {contents}%目录标题格式
}


% 设置图表题注的格式
\renewcommand{\thefigure}{\thechapter{}.\arabic{figure}}%把序号改为 (章序号.节序号.图序号)的格式
\renewcommand{\thetable}{\thesection{}.\arabic{table}}
\setlength {\belowcaptionskip} {-10pt}%设置标题与下方正文的垂直间距
\captionsetup {
  format = hang,%悬挂缩进
  font = footnotesize,%字号
  labelfont = bf,%加黑
  labelsep = quad,%分隔符为一个空格
  skip = 8pt,%标题与图表内容的垂直间距
}

%定理环境格式
{
  \theoremstyle{plain}
  \theorembodyfont{\upshape}
  \theoremsymbol{\hfill$\square$\par}
  % \theoremsymbol{\hfill\ensuremath{a\atop b}}
  % \theoremsymbol{\rule{1ex}{1ex}}
  \newtheorem{proof}{proof~}
  \newtheorem{theorem}{theorem}[chapter]
  \newtheorem{lemma}{lemma}[chapter]
  \newtheorem{corollary}{corollary}[chapter]
  \newtheorem{example}{example}[chapter]
  \newtheorem{answer}{answer}[chapter]
  \newtheorem{note}{note}[chapter]
}



%修改脚注编号为带圈数字
\renewcommand{\thefootnote}{\ding{\numexpr171+\value{footnote}}}

%自定义习题环境
\newenvironment{exprcise}[1][Exercise]{\begin{center}\bfseries\zihao{-3} #1\end{center}}{}

%修改列表项的编号
\renewcommand{\theenumi}{\thesection.\arabic{enumi}}
\renewcommand{\theenumii}{\alph{enumii}}
\renewcommand{\labelenumi}{\theenumi}
\setlist[enumerate,1]{itemsep=2pt, parsep=2pt}
\setlist[enumerate,2]{itemsep=1pt, parsep=1pt, topsep=0pt, }


\newcommand{\daochu}[1]{$G[\{ #1 \}]$}%为了解决1.4节插图时子图标题的问题



\begin{document}

  % 封面
  \pagestyle{empty}


  \frontmatter
  % 目录
  \tableofcontents
  % 目录的页眉页脚
  \pagestyle{empty}
  \pagestyle{fancy}
  \fancyhf{}%清除所有页眉页脚
  \fancyhead[OL]{\zihao{-5}contents}
  \fancyhead[OR]{$\cdot$~\thepage~$\cdot$}
  \fancyhead[EL]{$\cdot$~\thepage~$\cdot$}
  \fancyhead[ER]{\zihao{-5}contents}
  \clearpage{\pagestyle{empty}\cleardoublepage}
  
  \mainmatter
  % 正文的页眉页脚
  \fancyhead[OL]{\zihao{-5} \nouppercase\leftmark}
  \fancyhead[ER]{\zihao{-5} \nouppercase\rightmark}

\chapter{轴向拉伸和压缩}
\begin{enumerate}[label={\arabic*}] 
    \item 正应力与切应力\\
    正应力$\sigma$:应力与截面垂直的法向分量(离开截面为正)

切应力$\tau $:应力与截面相切的切向分量(对截面
内部(靠近截面)的点产生顺时针转向力矩的为正)

  \item 斜截面上的应力:\\
  
\begin{align}
    \sigma_\alpha=\frac{F}{A/\cos \alpha}\cos \alpha=\sigma\cos^2 \alpha \notag\\
    \tau_\alpha=\frac{F}{A/\cos \alpha}\sin \alpha=\sigma\sin\alpha\cos\alpha \notag
\end{align}

\item 拉压杆的变形与胡克定律
\begin{itemize}
    \item 1、纵向线应变:$\varepsilon =\frac{\delta l}{l}$;2、横向线应变:$\varepsilon' =\frac{\delta d}{d}$;3、泊松比:$\nu =|\frac{\varepsilon '}{\varepsilon }|$
    \item $\delta l=\frac{F_NL}{EA}$
    \item $\varepsilon =\frac{\sigma }{E}$
\end{itemize}

\item 应变能
\begin{itemize}
    \item 应变能:$V_\varepsilon =\frac{F_N^2l}{2EA}$
    \item 应变能密度:$v_\varepsilon=\frac{\sigma ^2}{2E}=\frac{E\sigma ^2}{2}$
    \item 功能原理:在弹性体变形的过程中,积蓄在弹性体内的应变能$V_\varepsilon $数值上等于外力做功
\end{itemize}
\item 材料的力学性能
\begin{enumerate}
    \item 标准试样\\
    圆截面:$l_0=5d_0 or l_0=10d_0$\\
    矩形截面:$l_0=5.65\sqrt{A_0} or l_0=11.3\sqrt{A_0}$
    \item 各变形阶段
    \begin{itemize}
        \item 弹性阶段:\\
        弹性变形:外力卸去后能够恢复的变形;塑性变形(永久变形):外力卸去后不能恢复的变形\\
        a:$\sigma_e$,弹性极限,卸载后不发生塑性变形的极限\\
        b:$\sigma_p$,比例极限,应力与应变符合胡克定律的最高限\\
        \item 屈服阶段:\\
        上屈服强度:在屈服阶段内,发生屈服应力首次下降前所对应的最高应力\\
        下屈服强度(屈服极限)$\sigma_s$:不计初识瞬时效应时的最低应力\\
        \item 强化阶段:\\
        主要为塑性变形\\
        强度极限$\sigma_b$,试样中名义应力的最大值\\
        \item 局部变形阶段:\\
        出现“缩颈”现象
    \end{itemize}
    衡量材料强度的指标,$\sigma_s,\sigma_b$
    \begin{figure}[htbp]
        \centering
        \includegraphics[width=0.5\textwidth]{1-3.png}
      \end{figure}
    \item 铸铁与低碳钢
    \begin{figure}[htbp]
        \centering
        \includegraphics[width=0.5\textwidth]{1-4.png}
      \end{figure}
      \begin{figure}[htbp]
        \centering
        \includegraphics[width=0.5\textwidth]{1-5.png}
      \end{figure}
\end{enumerate}
\item 拉(压)杆的强度计算
\begin{itemize}
    \item 对于塑性材料:$\sigma_u=\sigma_s$
    \item 对于脆性材料:$\sigma_u=\sigma_b$
    \item 许用应力:$[\sigma]=\frac{\sigma_u}{n}$
\end{itemize}
\item 应力集中\\
因杆件外形突然变化而引起局部应力急剧增大的现象

理论应力集中因数:$K=\frac{\sigma_{max}}{\sigma_{nom}}$
\begin{itemize}
    \item $\sigma_{max}$,发生应力集中的截面上的最大应力
    \item $\sigma_{nom}$,同一截面上按净面积算出的平均应力
\end{itemize}
\end{enumerate}

\newpage

\chapter{扭转}
\begin{enumerate}[label={\arabic*}] 
  \item 薄壁圆筒的扭转
\end{enumerate}

\chapter{组合变形及连接部分的计算}
组合变形:构件几种变形所对应的应力属于同一数量级
\begin{enumerate}[label={\arabic*}] 
    \item 两相互垂直平面内的弯曲
   
    \begin{itemize}
        \item $\sigma=\frac{M_y}{I_y}z-\frac{M_z}{I_z}y$
        \item 中性轴方程:$\frac{M_y}{I_y}z_0-\frac{M_z}{I_z}y_0=0$
    \end{itemize}
  \item 拉(压)与弯曲的组合
  
  \end{enumerate}

\end{document}