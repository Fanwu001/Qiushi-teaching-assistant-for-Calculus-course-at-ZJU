
%\documentclass[12pt]{article}
\documentclass[12pt]{scrartcl}
\title{Note01}
\nonstopmode
%\usepackage[utf-8]{inputenc}
\usepackage{graphicx} % Required for including pictures
\usepackage[figurename=Figure]{caption}
\usepackage{float}    % For tables and other floats
\usepackage{verbatim} % For comments and other
\usepackage{amsmath}  % For math
\usepackage{amssymb}  % For more math
\usepackage{fullpage} % Set margins and place page numbers at bottom center
\usepackage{paralist} % paragraph spacing
\usepackage{listings} % For source code
\usepackage{subfig}   % For subfigures
%\usepackage{physics}  % for simplified dv, and 
\usepackage{enumitem} % useful for itemization
\usepackage{siunitx}  % standardization of si units

\usepackage{tikz,bm} % Useful for drawing plots
%\usepackage{tikz-3dplot}
\usepackage{circuitikz}
\usepackage[UTF8]{ctex}

%%% Colours used in field vectors and propagation direction
\definecolor{mycolor}{rgb}{1,0.2,0.3}
\definecolor{brightgreen}{rgb}{0.4, 1.0, 0.0}
\definecolor{britishracinggreen}{rgb}{0.0, 0.26, 0.15}
\definecolor{cadmiumgreen}{rgb}{0.0, 0.42, 0.24}
\definecolor{ceruleanblue}{rgb}{0.16, 0.32, 0.75}
\definecolor{darkelectricblue}{rgb}{0.33, 0.41, 0.47}
\definecolor{darkpowderblue}{rgb}{0.0, 0.2, 0.6}
\definecolor{darktangerine}{rgb}{1.0, 0.66, 0.07}
\definecolor{emerald}{rgb}{0.31, 0.78, 0.47}
\definecolor{palatinatepurple}{rgb}{0.41, 0.16, 0.38}
\definecolor{pastelviolet}{rgb}{0.8, 0.6, 0.79}
\begin{document}

\begin{center}
	\hrule
	\vspace{.4cm}
	{\textbf { \large 云峰朋辈辅学微甲提升2组 --- 第3讲}}
\end{center}
{\textbf{内容提要:}\ 傅里叶级数 \hspace{\fill} \textbf{Date:} April 3 2022    \\
{ \textbf{主讲人:}} \ Famiglisti @CC98  \hspace{\fill} \textbf{Place:} 碧2党员之家 \\
	\hrule
~\\
\begin{center}
\textbf{\large{数学主要的目标是公众的利益和自然现象的解释。--傅里叶} }
\end{center}  
\paragraph*{\large 1 知识概要}
\begin{enumerate}
    \item  \textbf{傅里叶分析之开创性}\\
    \hspace*{0.7cm}任何周期函数,都可以看作是不同振幅,不同相位正弦波的叠加。就像,
    利用对不同琴键不同力度,不同时间点的敲击,可以组合出任何一首乐曲。
    \begin{figure}[htbp]
        \centering
        \includegraphics[width=12cm]{p1.png}
        \end{figure}
    
    \hspace*{0.7cm}除了纯数学之外,主要的受益者不是热力学,而是工程学,特别是电子工程。
    傅里叶变换已经成为科学和工程中的常规工具;它的应用包括从声音记录中去除噪音;利用x射线衍
    射发现DNA等大型生物化学分子的结构;改善无线电接收,处理从空中拍摄的照片。
    \begin{figure}[htbp]
        \centering
        \includegraphics[width=10cm]{p2.png}
        \end{figure}
    \item \textbf{幂级数与傅里叶级数}\\
    \hspace*{0.7cm}都用于将复杂函数转换成熟悉、易分析的简单函数,幂级数的局限性在于,
    用Taylor级数部分和近似代替函数f(x)时,要求f(x)至少有n阶的导数,且
    一般来说Taylor多项式仅在$x_0$附近与f(x)吻合得较为理想。与Taylor展开相比,Fourier展开对于f(x)
    的要求要宽容很多,并且它的部分和在整个区间都与f(x)吻合得较为理想。\\
    \item  \textbf{三角函数系及正交性}\\
    \hspace*{0.7cm}傅立叶分析之所以有效,是因为它的基本波形既正交又完整,
    而且如果适当地叠加,它们足以表示任何信号。\\
    三角函数系:\quad $1,\cos x,\sin x,\cos 2x,\sin 2x,\cos 3x,\sin 3x,\cos 4x,\sin 4x\cdots \cdots $
    \begin{enumerate}
        \item 
            $\int_{-\pi }^{\pi } \cos mx\cos nx \,dx= $
            $\left\{
            \begin{array}{lr}
            2\pi,m=n=0, &  \\
            \pi,m=n\geq 1,\\
            0,m\neq n, &  
            \end{array}
            \right.$
        \item 
            $\int_{-\pi }^{\pi } \sin mx\sin nx \,dx= $
            $\left\{
            \begin{array}{lr}
            \pi,m=n, &  \\
            0,m\neq n, &  
            \end{array}
            \right.$
    \end{enumerate}
~\\
    \item  \textbf{狄利克雷充分条件}\\
    设函数f(x)是以$2\pi$ 为周期的函数,若f(x)在$[-\pi,\pi]$上满足:
    \begin{enumerate}
        \item f(x)在$[-\pi,\pi]$上连续或只有有限个第一类间断点
        \item f(x)在$[-\pi,\pi]$上上只有有限个极值点
    \end{enumerate}
    则函数f(x)可展开成三角级数$ \frac{a_0}{2}+\sum_{n = 1}^{\infty} (a_n \cos nx+b_n\sin nx)  $,
    其中系数计算公式为
    \begin{gather}
        a_0=\frac{1}{\pi}\int_{-\pi}^{\pi}f(x) \,dx \notag\\
        a_n=\frac{1}{\pi}\int_{-\pi}^{\pi}f(x)\cos nx \,dx \notag\\
        b_n=\frac{1}{\pi}\int_{-\pi}^{\pi}f(x)\sin nx \,dx \notag
    \end{gather}
    级数的收敛域为$(-\infty,\infty)$\\
    级数$ \frac{a_0}{2}+\sum_{n = 1}^{\infty} (a_n \cos nx+b_n\sin nx)  $与f(x)的关系:\\
    \begin{enumerate}
        \item 当x为f(x)的连续点时,$ \frac{a_0}{2}+\sum_{n = 1}^{\infty} (a_n \cos nx+b_n\sin nx) =f(x) $
        \item 当x为f(x)的间断点时,$ \frac{a_0}{2}+\sum_{n = 1}^{\infty} (a_n \cos nx+b_n\sin nx) =\frac{f(x-0)+f(x+0)}{2} $
    \end{enumerate}
~\\
    \item  \textbf{间断点}\\
    函数连续的定义是什么?
      \begin{enumerate}
          \item 第一类间断点: $f(x_0+)\neq f(x_0-)$ 
          \item 第二类间断点: 左右极限至少有一个不存在
          \item 第三类间断点:$f(x_0+)=f(x_0-),f(x_0)无意义$
      \end{enumerate}
~\\
    \item  \textbf{定义于$[0,\pi]$上f(x)的正弦级数与余弦级数}\\
    step1:\quad f(x)奇延拓or偶延拓F(x)\\
    step2:\quad 将F(x)在$[-\pi,\pi]$上展成傅里叶级数
    \begin{enumerate}
        \item 奇延拓展成正弦级数\\
        \begin{gather}
            a_0=0 \notag\\
            a_n=0 \notag\\
            b_n=\frac{2}{\pi}\int_{0}^{\pi}f(x) \sin{nx} \,dx \notag
        \end{gather}
        正弦级数为$ \sum_{n = 1}^{\infty} b_n \sin nx  $
        \item 偶延拓展成余弦级数
        \begin{gather}
            a_0=\frac{2}{\pi}\int_{0}^{\pi}f(x) \,dx \notag\\
            a_n=\frac{2}{\pi}\int_{0}^{\pi}f(x)\cos nx \,dx \notag\\
            b_n=0 \notag\\
        \end{gather}
        余弦级数为$ \frac{a_0}{2}+\sum_{n = 1}^{\infty} a_n \cos nx  $
    \end{enumerate}
    \newpage
    \item  \textbf{一些变形}
    \begin{enumerate}
        \item 周期为2l的函数的傅里叶级数
        \item 定义于$[-l,l]$上的函数f(x)的傅里叶级数
        \item 定义于$[0,l]$上的函数f(x)的正弦级数与余弦级数
    \end{enumerate}
~\\
    \item \textbf{Parseval等式}\\
    \begin{gather}
        f(x)\sim \frac{a_0}{2}+\sum_{n = 1}^{\infty} (a_n \cos nx+b_n\sin nx)  \notag\\
        \frac{a_0^2}{2}+\sum_{n = 1}^{\infty} (a_n^2+b_n^2)= \frac{1}{\pi}\int_{-\pi}^{\pi}f^2(x)  \,dx   \notag
    \end{gather}
    成立条件,f在$[-\pi,\pi]$内平方可积,推广形式:
    \begin{gather}
        \frac{a_0\alpha _0}{2}+\sum_{n = 1}^{\infty} (a_n\alpha _n+b_n\beta _n)= \frac{1}{\pi}\int_{-\pi}^{\pi}f(x)g(x)  \,dx   \notag
    \end{gather}
    成立条件,f,g在$[-\pi,\pi]$内平方可积
\end{enumerate}

\paragraph*{\large 2 例题}\leavevmode \newline
【example 1】 $f(x)=x^2\quad (0\leq x<1)$展成正弦级数级数,设其和函数
为S(x),求$S(-3),S(-\frac{15}{2} )$\\
\\
\\
\\
\\
\\
\\
【example 2】 将函数$f(x)=x^2\quad (-\pi<x<\pi)$展成Fourier级数,并求
$\sum_{n = 1}^{\infty} \frac{(-1)^n}{n^2}$  \\
\\
\\
\\
\\
\\
\\
【example 3】 将函数$f(x)=\left\lvert x\right\rvert  (-\pi\leq x\leq \pi)$展成Fourier级数,并求
$\sum_{n = 1}^{\infty} \frac{1}{n^2}$  \\
\\
\\
\\
\\
\\
\\
\paragraph*{\large 3 真题解析}\leavevmode \newline
【18-19mid】将函数$f(x)=e^x\quad (0\leq x< 2\pi)$展成周期为$2\pi$的Fourier级数 \leavevmode \newline
\\
\\
\\
\\
\\
\\
\\
【19-20final】利用傅里叶级数理论证明:$\forall x\in (0,\pi),
\sum_{n = 1}^{\infty} \frac{\sin nx}{n}=\frac{\pi-x}{2}    $\leavevmode \newline
\\
\\
\\
\\
\\
\\
\\
【20-21final】$f(x)=e^x\quad (0<x\leq 1)$,
将f延拓成$\mathbb{R} $上周期为2的函数,记为f,设其傅里叶级数
$ \frac{a_0}{2}+\sum_{n = 1}^{\infty} (a_n \cos nx+b_n\sin nx)  $
,求$\sum_{n = 1}^{\infty} (-1)^na_n ,\sum_{n = 1}^{\infty} a_n$\leavevmode \newline
\\
\\
\\
\\
\\
\\
\\
\\
\\
【20-21final】求$\sum_{n = 1}^{\infty}(-1)^n\frac{n}{n+\frac{1}{2} }x^{2n} $
的收敛半径、和函数\leavevmode \newline  
\\
\\
\\
\\
\\
\\
\\
\\
\\
\paragraph*{\large 4 级数综合}\leavevmode \newline
【ex1】设f(x)在$[ 0,+\infty]$上连续,且$\int_{0}^{+\infty}f^2(x) \,dx $收敛,令$a_n=\int_{0}^{1}f(nx)  \,dx $,证明
 $\sum_{n = 1}^{\infty} \frac{a_n^2}{n}  $收敛
\\
\\
\\
\\
\\
\\
\\
\\
\\
\\
【ex2】求$\sum_{n = 1}^{\infty} \frac{1}{n^4},\sum_{n = 1}^{\infty} \frac{1}{n^6}  $\leavevmode \newline
\\
\\
\\
\\
\\
\\
\\
\\
【ex3】求$1+\sum_{n = 1}^{\infty} \frac{(2n-1)!!}{(2n)!!x^n}$的和函数  
\\
\\
\\
\\
\\
\\
\\
\\
\paragraph*{\large 5 参考书目}\leavevmode \newline
\begin{enumerate}
    \item 陈纪修,於崇华,金路. 数学分析.下册[M]. 北京:高等教育出版社,2019.5
    \item 汤家凤.考研数学复习大全[M]. 北京:中国原子能出版社,2019.2
    \item 谢惠民.数学分析习题课讲义.下册[M]. 北京:高等教育出版社,2004.1
\end{enumerate}
\end{document}