%\documentclass[12pt]{article}
\documentclass[12pt]{scrartcl}
\title{Note01}
\nonstopmode
%\usepackage[utf-8]{inputenc}
\usepackage{graphicx} % Required for including pictures
\usepackage[figurename=Figure]{caption}
\usepackage{float}    % For tables and other floats
\usepackage{verbatim} % For comments and other
\usepackage{amsmath}  % For math
\usepackage{amssymb}  % For more math
\usepackage{fullpage} % Set margins and place page numbers at bottom center
\usepackage{paralist} % paragraph spacing
\usepackage{listings} % For source code
\usepackage{subfig}   % For subfigures
%\usepackage{physics}  % for simplified dv, and 
\usepackage{enumitem} % useful for itemization
\usepackage{siunitx}  % standardization of si units

\usepackage{tikz,bm} % Useful for drawing plots
%\usepackage{tikz-3dplot}
\usepackage{circuitikz}
\usepackage[UTF8]{ctex}


\begin{document}

\begin{center}
	\hrule
	\vspace{.4cm}
	{\textbf { \large 云峰朋辈辅学微甲提升2组 --- 第7讲}}
\end{center}
{\textbf{内容提要:}\ 三重积分 \hspace{\fill} \textbf{Date:} May 22 2022    \\
{ \textbf{主讲人:}} \ Famiglisti @CC98  \hspace{\fill} \textbf{Place:} 碧2党员之家 \\
	\hrule
~\\


\paragraph*{\large 1 preface}\leavevmode \newline

时间过得太快太快$\~$

本学期微积分的内容只剩下第一类曲线积分、第二类曲面积分,按照历年情况,
重积分、曲线、曲面积分在期末考卷中占有较大的分值:
\begin{itemize}
    \item 21.6 6题 50-60分,1三重积分,3曲线积分,2曲面积分
    \item 20.9 6题 50-60分,1二重积分,1三重积分,2曲线积分,2曲面积分
    \item 19.6 6题 50-60分,1二重积分,1三重积分,2曲线积分,1曲面积分,1二重积分与曲面积分结合
\end{itemize}

积分题目,涉及到大量的计算,容易出错,大家做题的时候仔细些,耐心一些,问题不大。
同时,对于积分计算中的常用技巧,如对称区域求积、变量替换法、极坐标法需
熟练掌握,由于这些内容在上一份讲义中以详细解说过,本次课程不再重复讲解。

另外,重积分的物理应用,质量、质心坐标、转动惯量怎么求也需要知道。

这次课程我们先来做一些历年考卷中涉及二重积分、三重积分的题目,熟悉一下应试难度,之后再
做一些积分相关证明题。
    

\paragraph*{\large 2 真题}\leavevmode \newline
【18-19final】试求三重累次积分$\int_{0}^{1}  \,dx\int_{0}^{1}  \,dy\int_{y}^{1} \frac{e^{-z^2}}{x^2+1} \,dz $
\\
\\
\newpage
【18-19final】设$\mathbb{R} ^3$中有一抛物面壳$z=\frac{1}{2}(x^2+y^2)(0\leq x \leq 1)$,
已知其面密度为正常数c,试求其重心坐标。
\\
\\
\\
\\
\\
\\
\\
\\
\\
\\
\\
\\
\\
【19-20final】求$\mathbb{R} ^3$中封闭曲线S由二元连续函数$\rho=\rho(\theta,\varphi ),(\theta,\varphi )\in[0,2\pi]\times[0,\pi] ,(\rho,\theta,\varphi)$为球坐标,证明:
S所围的有界闭立体$\Omega $的体积为
\begin{align}
    V(\Omega)=\frac{1}{3}\int_{0}^{2\pi}  \,d\theta\int_{0}^{\pi}[\rho(\theta,\varphi ),(\theta,\varphi)]^3\sin \varphi  \,d\varphi \notag  
\end{align}
\\
\\
\\
\\
\\
\\
\\
\\
\\
【19-20final】 求封闭曲面$(x^2+y^2)^2+z^4=y$所围的空间有界闭立体K的体积V(K)  
\\
\\
\newpage
【20-21final】 设$K=\left\{(x,y,z)\in \mathbb{R} ^3|x^2+y^2+z^2\leq z\right\} $,
计算$ \iiint_K(z+\sqrt{x^2+y^2+z^2}) dxdydz $
\\
\\
\\
\\
\\
\\
\\
\\

\paragraph*{\large 3 例题}\leavevmode \newline
【example 1】设f(u)为连续函数,$F(t)=\iiint_\Omega [z^2+f(x^2+y^2)] dv$,
其中$\Omega=\left\{(x,y,z)|0\leq z \leq h,x^2+y^2\leq t^2 \right\} $,求$\lim_{t \to 0^+}\frac{F(t)}{t^2}  $
\\
\\
\\
\\
\\
\\
\\
\\
\\
\\
\\
【example 2】设$f,\frac{\partial f}{\partial x},\frac{\partial f}{\partial t},\frac{\partial^2 f}{\partial x^2}$均为$[0,1]\times[0,1]$
中的连续函数,且在$[0,1]\times[0,1]$中成立$\frac{\partial f}{\partial t}=\frac{\partial^2 f}{\partial x^2}$和$\left\lvert \frac{\partial f}{\partial x}\right\rvert \leq 1 $,证明:
\begin{enumerate}
    \item 对任何$(x,t_1),(x,t_2)\in [0,1]\times[0,1]$,存在$\xi \in[0,1],s.t. |\xi-x|\leq \frac{1}{2}|t_1-t_2|$且$\left\lvert f(\xi,t_1)-f(\xi,t_2)\right\rvert \leq 4|t_1-t_2|^\frac{1}{2}$
    \item 由(1)的结论证明,对任何$(x,t_1),(x,t_2)\in [0,1]\times[0,1]$成立$\left\lvert f(x,t_1)-f(x,t_2)\right\rvert \leq 5|t_1-t_2|^\frac{1}{2}$
\end{enumerate}
提示:交换积分次序
\\
\\
\newpage
【example 3】若直线x=0,x=a,y=0与正连续曲线y=f(x)围成的区域的质心的x坐标是g(a),证明:
\begin{align}
    f(x)=\frac{Ag'(x)}{[x-g(x)]^2}exp(\int_{}^{} \frac{1}{x-g(x)} \,dx ) \notag
\end{align}
其中A为正常数,a是参数
\\
\\
\\
\\
\\
\\
\\
\\
\\
\\
\\
\\
\\
\\
\\
\paragraph*{\large 4 参考书目}\leavevmode \newline
\begin{enumerate}
    \item 陈纪修,於崇华,金路. 数学分析.下册[M]. 北京:高等教育出版社,2019.5
    \item 汤家凤.考研数学复习大全[M]. 北京:中国原子能出版社,2019.2
    \item 谢惠民.数学分析习题课讲义.下册[M]. 北京:高等教育出版社,2004.1
    \item 苏德矿,吴明华.微积分.下[M].北京:高等教育出版社,2007.7
\end{enumerate}
\end{document}