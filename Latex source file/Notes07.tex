%\documentclass[12pt]{article}
\documentclass[12pt]{scrartcl}
\title{Note01}
\nonstopmode
%\usepackage[utf-8]{inputenc}
\usepackage{graphicx} % Required for including pictures
\usepackage[figurename=Figure]{caption}
\usepackage{float}    % For tables and other floats
\usepackage{verbatim} % For comments and other
\usepackage{amsmath}  % For math
\usepackage{amssymb}  % For more math
\usepackage{fullpage} % Set margins and place page numbers at bottom center
\usepackage{paralist} % paragraph spacing
\usepackage{listings} % For source code
\usepackage{subfig}   % For subfigures
%\usepackage{physics}  % for simplified dv, and 
\usepackage{enumitem} % useful for itemization
\usepackage{siunitx}  % standardization of si units

\usepackage{tikz,bm} % Useful for drawing plots
%\usepackage{tikz-3dplot}
\usepackage{circuitikz}
\usepackage[UTF8]{ctex}


\begin{document}

\begin{center}
	\hrule
	\vspace{.4cm}
	{\textbf { \large 云峰朋辈辅学微甲提升2组 --- 第6讲}}
\end{center}
{\textbf{内容提要:}\ 二重积分 \hspace{\fill} \textbf{Date:} May 15 2022    \\
{ \textbf{主讲人:}} \ Famiglisti @CC98  \hspace{\fill} \textbf{Place:} 碧2党员之家 \\
	\hrule
~\\


\paragraph*{\large 1 知识概要}
\begin{enumerate}
    \item  \textbf{一些概念}
        \begin{itemize}
        \item 区域分割、区域的直径
        \item 区域的面积、零面积
        \item 黎曼积分vs勒贝格积分   
        \end{itemize}

    \item \textbf{二重积分的一些性质}
        \begin{itemize}
            \item 积分中值定理\\
            设f(x,y)在平面有限闭区域上连续,A为D的面积,则存在
            $(\xi ,\eta )\in D,s.t. \iint _Df(x,y)d\sigma =f(\xi ,\eta)A$
            \item 线性性、可加性、区域可加性
        \end{itemize}

    \item \textbf{计算技巧}\\
       \begin{enumerate}
           \item 对称区域求积
           \begin{itemize}
               \item 区域D关于y轴对称,右侧区域为$D_1$
               \begin{itemize}
                   \item 当f(-x,y)=f(x,y)时,$\iint _D f(x,y)dxdy=2\iint _{D_1} f(x,y)dxdy$
                   \item 当f(-x,y)=-f(x,y)时,$\iint _D f(x,y)dxdy=0$
               \end{itemize}
               \item 区域D关于x轴对称,上侧区域为$D_1$
               \begin{itemize}
                   \item 当f(x,-y)=f(x,y)时,$\iint _D f(x,y)dxdy=2\iint _{D_1} f(x,y)dxdy$
                   \item 当f(x,-y)=-f(x,y)时,$\iint _D f(x,y)dxdy=0$
               \end{itemize}
               \item 区域D关于y=x对称,$\iint _D f(x,y)dxdy=\iint _D f(y,x)dxdy$
               \item 区域D关于y=-x对称,$\iint _D f(x,y)dxdy=\iint _D f(-y,-x)dxdy$
           \end{itemize}
           \item 坐标变换
           \begin{itemize}
               \item 变量替换公式推导\\
               设$x=x(u,v),y=y(u,v),(u,v)\in D'$\\
               这一代换满足:
               \begin{enumerate}
                   \item 建立了D与D'之间的一一对应;
                   \item x,y在D'内具有各个变元的连续偏导数,并且其
                   逆变换u=u(x,y),v=v(x,y)在D内也具有各个变元的连续偏导数
                   \item 代换的Jacobi行列式$J=\frac{\partial(x,y)}{\partial(u,v)}$在D'内无零点
               \end{enumerate}
               则,$\iint_Df(x,y)dxdy=\iint _D'f(x(u,v),y(u,v))|\frac{\partial(x,y)}{\partial(u,v)}|dudv$ 
               \item 极坐标变换\\
               $\iint _Df(x,y)d\sigma =\int_{\alpha}^{\beta}  \,d\theta\int_{r_1(\theta)}^{r_2(\theta)} rf(r\cos \theta,r\sin \theta) \,dr  $
               用极坐标计算二重积分一般至少需要满足如下两个特征之一:
               \begin{itemize}
                   \item 积分区域的边界曲线含$x^2+y^2$
                   \item 被积函数f(x,y)的表达式中含$x^2+y^2$
               \end{itemize}
           \end{itemize}
           \item x型区域与y型区域
           \begin{figure}[htbp]
            \centering
            \includegraphics[width=10cm]{1.png}
            \end{figure}
       \end{enumerate}  
    \item \textbf{应用}\\
    设$\sum :z=\varphi (x,y)((x,y)\in D)$为空间曲面,则该曲面段的面积为:
    \begin{align}
        A=\iint _D\sqrt{1+(\frac{\partial z}{\partial x})^2+(\frac{\partial z}{\partial y})^2}d\sigma \notag
    \end{align}

    
\end{enumerate}
\newpage
    

\paragraph*{\large 2 例题}\leavevmode \newline
【example 1】设$l:x=\varphi (t),y=\psi(t),\alpha\leq t\leq \beta  $
$\varphi,\psi$连续,且至少其中之一有连续导数,则曲线l的面积为零。
\\
\\
\\
\\
\\
\\
\\
\\
\\
\\
\\
【example 2】求由曲线$(\frac{x^2}{a^2}+\frac{y^2}{b^2})^2=\frac{x^2}{a^2}-\frac{y^2}{b^2}$所围的面积。
\\
\\
\\
\\
\\
\\
\\
\\
\\
\\
\\
\\
\\
【example 3】求$\iint (\sqrt{\frac{x-c}{a}}+\sqrt{\frac{y-c}{b}})dxdy$,
其中D由曲线$\sqrt{\frac{x-c}{a}}+\sqrt{\frac{y-c}{b}}=1,x=c,y=c$围成,且a,b,c>0
\\
\\
\\
\\
\\
\\
\\
\\
\\
【example 4】 求$I=\iint _\varOmega (x+y)dxdy$,其中$\varOmega $由
$y^2=2x,x+y=4,x+y=12$围成  \\
\\
\\
\\
\\
\\
\\
\\
\\
\\
【example 5】 设$D:x^2+y^2\leq 4$,则$\iint_D(x-2y)^2 d\sigma=$  \\
\\
\\
\\
\\
\\
\\
\\
\\
\\
【example 6】设$D:x^2+y^2\leq 1(x\geq 0,y\geq 0)$,则$\iint_D\frac{x^2\sin(x^2+y^2)}{x^2+y^2} d\sigma=$  \\
\\
\\
\\
\\
\\
\\
\\
\\
\\
\newpage
\paragraph*{\large 5 参考书目}\leavevmode \newline
\begin{enumerate}
    \item 陈纪修,於崇华,金路. 数学分析.下册[M]. 北京:高等教育出版社,2019.5
    \item 汤家凤.考研数学复习大全[M]. 北京:中国原子能出版社,2019.2
    \item 谢惠民.数学分析习题课讲义.下册[M]. 北京:高等教育出版社,2004.1
    \item 苏德矿,吴明华.微积分.下[M].北京:高等教育出版社,2007.7
\end{enumerate}
\end{document}